% document class
\documentclass[journal,12pt,onecolumn,draftclsnofoot]{IEEEtran}

%%%%%%%%%%%%%%%%%%%%%%%%%%%%%%%%%%%%
% packages and formatting stuff
%%%%%%%%%%%%%%%%%%%%%%%%%%%%%%%%%%%%

%\usepackage{cite}
\usepackage{setspace}
\usepackage{subcaption}
\usepackage{xcolor}
\usepackage{amsmath}
\usepackage{amssymb} %more math
\usepackage{float} % force images to be where you actually put them
\usepackage{pdfpages} % add pdfs
\usepackage[most]{tcolorbox} % frames around images


%%%%%%%%%%%%%%%%%%%%%%
%%% inserting code %%%
%%%%%%%%%%%%%%%%%%%%%%
\usepackage{listings}
\usepackage{color}

\definecolor{mygreen}{rgb}{0,0.6,0}
\definecolor{mygray}{rgb}{0.5,0.5,0.5}
\definecolor{mymauve}{rgb}{0.58,0,0.82}
\lstset{ %
	frame=none,
	backgroundcolor=\color{white},   % choose the background color
	basicstyle=\footnotesize,        % size of fonts used for the code
	breaklines=true,                 % automatic line breaking only at whitespace
	captionpos=b,                    % sets the caption-position to bottom
	commentstyle=\color{mygreen},    % comment style
	escapechar=\%,					 % if you want to add LaTeX within your code
	keywordstyle=\color{blue},       % keyword style
	stringstyle=\color{mymauve},     % string literal style
	showstringspaces=false,
}
%%%%%%%%%%%%%%%%%%%%%%%%%%
%%% BIBLIOGRAPHY STUFF %%%
%%%%%%%%%%%%%%%%%%%%%%%%%%
\usepackage[backend=biber,style=ieee,urldate=iso,date=iso,seconds=true]{biblatex}
\bibliography{eece_resources.bib}

%%%%%%%%%%%%%%%%%%%%%%
%%% PAGE FORMATING %%%
%%%%%%%%%%%%%%%%%%%%%%
\addtolength{\oddsidemargin}{-1.cm}
\addtolength{\textwidth}{2cm}
\addtolength{\topmargin}{-2cm}
\addtolength{\textheight}{3.5cm}

\def\changemargin#1#2{\list{}{\rightmargin#2\leftmargin#1}\item[]}
\let\endchangemargin=\endlist

\usepackage[margin=1in]{geometry}

\renewcommand\IEEEkeywordsname{}
\renewcommand\abstractname{}
\newcommand{\HRule}{\rule{\linewidth}{0.5mm}}

\usepackage[margin=1in]{geometry}
\linespread{1.1} %  linespacing
\renewcommand\IEEEkeywordsname{}
\renewcommand\abstractname{}

% correct bad hyphenation here
\hyphenation{op-tical net-works semi-conduc-tor}


%%%%%%%%%%%%%%%
%%% MACROS %%%%
%%%%%%%%%%%%%%%
% adding figures (path, caption, label)
% eg. \pic{path/to/image.png}{some nice caption text}{fig:image_example}

\newcommand{\pic}[3]{
	\vspace{2.5mm}
	\begin{center}
		\tcbox[sharp corners, boxrule=0.5mm, boxsep=0mm, colframe=black, colback=white, width=0.7\textwidth]{\includegraphics[width=0.85\textwidth]{#1}}
		\captionof{figure}{#2}
		\label{#3}
	\end{center}
	\vspace{2.5mm}
	}

%images next to each other
% eg.
%	\vspace{2.5mm}
%	\begin{figure}
%		\subpic{path/to/image.png}{some nice caption text}{fig:image_example}
%		\subpic{path/to/image2.png}{some nice caption text}{fig:image2_example}
%	\end{figure}
%	\vspace{2.5mm}

\newcommand{\subpic}[3]{
	\begin{subfigure}[h]{0.5\textwidth}
		\centering
		\tcbox[sharp corners, boxrule=0.5mm, boxsep=0mm, colframe=black, colback=white, width=0.7\textwidth]{\includegraphics[width=\textwidth]{#1}}
		\caption{#2}
		\label{#3}
	\end{subfigure}
	}

%smaller oscilloscope images
% eg. \oscpic{path/to/image.png}{some nice caption text}{fig:image_example}

\newcommand{\oscpic}[3]{
	\vspace{2.5mm}
	\begin{center}
		\tcbox[sharp corners, boxrule=0.5mm, boxsep=0mm, colframe=black!, colback=white, width=0.7\textwidth]{\includegraphics[width=0.7\textwidth]{#1}}
		\captionof{figure}{#2}
		\label{#3}
	\end{center}
	\vspace{2.5mm}
	}

%inserting python
	% eg. \inputpython{path/to/code.py}{caption about the script}{code:label}
	
\newcommand{\inputpython}[3]{
	\vspace{2.5mm}
	\lstinputlisting[language=python,caption=#2,label=#3]{#1}
	\vspace{2.5mm}
	}

%%%%%%%%%%%%%%%%%%%%%%%%%%%%%%%%%%%%
% the actual document
%%%%%%%%%%%%%%%%%%%%%%%%%%%%%%%%%%%%

\begin{document}

	%%%%%%%%%%%%%%%%%%%%%%%%%%%%%%%%%%%%%%%%%%%%%%%%%%%%%
	% Coverpage and plagiarism statement (use .pdf names)
	%%%%%%%%%%%%%%%%%%%%%%%%%%%%%%%%%%%%%%%%%%%%%%%%%%%%%

	\begin{titlepage}

	\begin{center}
		%the above is justification
		% Upper part of the page
		\includegraphics[width=1\textwidth]{./up-logo.jpg}\\[0.4cm]
		\textsc{\LARGE Department of Electrical, Electronic and Computer Engineering}\\[1.5cm]
		\textsc{\Large EMS 310 }\\[0.5cm]
		% Title
		\HRule \\[0.4cm]
		{ \huge \bfseries Practical 2: FM Modulation and Demodulation}\\[0.4cm]
		\HRule \\[0.4cm]


		\begin{minipage}{0.4\textwidth}
			\begin{flushleft} \large
				\emph{Author:}\\
			\end{flushleft}
		\end{minipage}
		\begin{minipage}{0.4\textwidth}
			\begin{flushright} \large
				\emph{Student number:} \\

			\end{flushright}
		\end{minipage}


		% Student Credentials
		\begin{minipage}{0.4\textwidth}
			\begin{flushleft} \large
				%\emph{Author:}\\
				\textsc{Kevin Nel}\\ % Put the name of the  Student here
			\end{flushleft}
		\end{minipage}
		\begin{minipage}{0.4\textwidth}
			\begin{flushright} \large
				17003769 %put Student Number here
			\end{flushright}
		\end{minipage}




		% Student Credentials
		\begin{minipage}{0.4\textwidth}
			\begin{flushleft} \large
				\textsc{Tiehan Nel} % Put the name of the  Student here
			\end{flushleft}
		\end{minipage}
		\begin{minipage}{0.4\textwidth}
			\begin{flushright} \large
				17330999 %put Student Number here
			\end{flushright}
		\end{minipage}


		% Student Credentials
		\begin{minipage}{0.4\textwidth}
			\begin{flushleft} \large
				\textsc{Anro v.d. Merwe} % Put the name of the  Student here
			\end{flushleft}
		\end{minipage}
		\begin{minipage}{0.4\textwidth}
			\begin{flushright} \large
				17004812 %put Student Number here
			\end{flushright}
		\end{minipage}


		\vfill
		% Bottom of the page
		{\large \today}
	\end{center}
\end{titlepage}


	\footnotesize

	\input{declaration_of_originality.tex}
	%\includepdf[pages=-]{signatures.pdf} % add scanned in signature page

	\normalsize

	\clearpage

	\thispagestyle{empty}
	\tableofcontents

	\clearpage

	\setcounter{page}{1} %sets page counter only after toc


	\title{Phase Locked Loop}

	% authors
	%Put in Student credentials here as well
	\author{Tiehan~Nel~\IEEEmembership{12345678},
			Anro~v.d.~Merwe~\IEEEmembership{12345678},
			Kevin~Nel~\IEEEmembership{17003769}% <-this % stops a space


	\thanks{Tiehan~Nel, Anro~v.d.~Merwe, and Kevin~Nel are EMS 310 students in group 5 with the Department of Electrical, Electronic and Computer Engineering at the University of Pretoria.}}% <-this % stops a space


	% The paper headers
	\markboth{Phase Locked Loop}%
		{Shell \MakeLowercase{\textit{et al.}}: Bare Demo of IEEEtran.cls for IEEE Journals}

	\maketitle


	%%%%%%%%%%%%%%%%%%%%%%%%%%%%%%
	% abstract and keywords
	%%%%%%%%%%%%%%%%%%%%%%%%%%%%%%

	\begin{abstract}
		\begin{changemargin}{-8.55mm}{-8.55mm}
		\textbf{\textit{Abstract}}---Abstract goes here.
	\end{changemargin}
	\end{abstract}



	\begin{IEEEkeywords}
		\vspace{-2.25em}
		\begin{changemargin}{-8.55mm}{-8.55mm}
			\textbf{\textit{Keywords}}---keywords
		\end{changemargin}
	\end{IEEEkeywords}




%%%%%%%%%%%%%%%%%%%%%%%%%%%%%%
% the real stuff
%%%%%%%%%%%%%%%%%%%%%%%%%%%%%%

	% place sections here
	\input{Introduction.tex}
	\clearpage
	\input{Aim.tex}
	\clearpage
	\input{Theoretical_Analysis.tex}
	\clearpage
	\input{Design.tex}
	%use the macros provided inmain.tex to add images:
% eg.
% \pic{path/to/image.png}{some caption tex}{fig:figure_label} 
% 
% note:
% when referencing images in text:
% ... as seen in figure~\ref{fig:figure_label}~
%

\section{Simulations}
	
	
	

	\clearpage
	\section{Results}
% use same method of adding pictures as in Simulations.tex

	\clearpage
	\input{Discussion.tex}
	\clearpage
	\input{Conclusion.tex}



\printbibliography

%%%%%%%%%%%%%%%%%%%%%%%%%%%%%%
% Appendices
%%%%%%%%%%%%%%%%%%%%%%%%%%%%%%
\clearpage
\appendices
\input{Appendix.tex}

\end{document}
